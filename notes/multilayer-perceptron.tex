\documentclass[letterpaper,12pt]{article}
\usepackage{amssymb,latexsym,amsmath}
\usepackage{wasysym}
\usepackage{graphicx}
\usepackage{hyperref}
\hypersetup{
    colorlinks=true,
    linkcolor=blue,
    filecolor=magenta,
    urlcolor=cyan,
}

\begin{document}

\begin{flushleft}
    Perceptrons\\
    AI/ML Guild Lab \#2\\
    \today \\
\end{flushleft}

\section{Multilayer Perceptrons}

It's really a misnomer to call Multilayer Perceptrons \emph{multi-layer}. It's
not a perceptron with multiple layers but instead it is a network of multiple
``neuron-like'' processing units but not every layer in the network is a
perceptron.

Recall that the perceptron equation:

\[
    Z = \sum_{i=0}^{N} W_i x_i + b
\]

What would happen if you applied multiple layers of this equation?

\section{History}

Dudes named Rumelhart, Hinton, and Williams introduced/discovered this algorithm
in the mid-‘80s. Rumelhart was a mathematical pychologist working at UC San
Diego. He led a research group on neural networks, picking up where Rosenblatt
left off after the Minksy book pretty much destroyed public funding for this
type of research.

Rumelhart wanted to train a neural network with multiple
layers and sigmoidal units instead of threshold units (as in the perceptron)
or linear units but they did not how to do that. Rumelhart came up with an idea
called ``Backpropagation'' but it was Hinton that used it in conjunction with
the training model Rumelhart developed for sigmoidal multilayered perceptrons.

This is what is typically referred to as a multilayered perceptron today.

\section{The Algorithm}

Formally, a multilayered perceptron is one where these conditions are true:

\begin{itemize}
    \item a linear function that aggregates the input values
    \item a sigmoid function (the activation function)
    \item an output function (i.e.\ a threshold function for classification)
    \item a loss function that computes the overall error of the network
    \item a learning procedure to adjust the weights of the network, i.e., the
        ``backpropagation'' algorithm
\end{itemize}

\section{The Math}

\subsection{Linear Function}
The linear function is very similar to the single layer perceptron. It is a
weight sum of the inputs plus a bias. The difference is that the weights are
matrix and no longer a vector.
\[
    z_m = \sum_{i=0}^{N} W_{m,n} x_m + b
\]

\subsection{Sigmoid Function}
Every element of the $Z_m$ becomes an input to the sigmoid function.

\[
    a_m = \sigma(z_m) = \frac{1}{1 + e^{-z_m}}
\]

The output of the sigmoid function is a vector of size $m$ where every element
is a unit in the hidden layer.

$a$ stands for activation! Which is typically how the output of the hidden layer
is described. Using the sigmoid funciton is somewhat arbrary, you can use other
non-linear functions here like $\tanh$ or $ReLU$.

\subsection{Output Function}

Similar to single layer but adjusted for sigmoid.

\[
    \sigma(a) =
    \begin{cases}
        1, & a \ge 0.5\\
        0, & \text{else}
    \end{cases}
\]

\subsection{Cost function}

We need a measure of how well the network is performing. We did this with the
single layer too. This is used in the training step. This often goes by
different names in the literature such \emph{objective function}, \emph{loss
function}, or \emph{error function}.

Typically this cost function is tuned for your problem at hand. The one that
Rumelhart came up with was

\[
    E = \frac{1}{2} \sum_{k} {(a_k - y_k)}^2
\]

This is commonly known as the sum of squared errors function. It's pretty basic
but often gets the job done.

\subsection{Training: Forward \& Back Propagation}

Multilayer Perceptron training is composed of two steps. Forward propagation
where information flows forward in the network to compute predictions and the
error. And then the back propagation stage where the error derivatives is
calculated and used to update the weight matrix.

The forward propagation stage is just chaining the linear, sigmoid and output
functions (just like the single layer!).

The back propagation part is more complicated but it's basically asking ``how
does the error change if we tweak the weights''? This question has additional
questions that need answers first. How does the error change when we change
the activation $a_m$ a bit?  How does the activation change if we fiddle with
linear function a bit?

It takes a fair amount of calculus to derive all of this. I'll leave it as an
excercise to the reader. The outcome of that derivation can be solved for using
the gradient descent algorithm.

\end{document}

