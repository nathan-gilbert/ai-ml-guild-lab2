\documentclass[letterpaper,12pt]{article}
\usepackage{amssymb,latexsym,amsmath}
\usepackage{wasysym}
\usepackage{hyperref}
\hypersetup{
    colorlinks=true,
    linkcolor=blue,
    filecolor=magenta,
    urlcolor=cyan,
}

\begin{document}

\begin{flushleft}
Perceptrons\\
AI/ML Guild Lab \#2\\
\today \\
\end{flushleft}

\section{Introduction}

The \emph{Perceptron} is a linear discrimant model which has an important place
in Machine Learning as being the first ``neural network''-style algorithm.

Neural Networks, which are pattern recognition systems somewhat based off of
biological neurons, are incredibly popular today. Perceptrongs were created by
Frank Rosenblatt, who was a phsychologist attempting to create a mathematical
model of biological neurons (special cells found in brains.)

\section{The \emph{perceptron} model}

There are $n$ binary inputs given as a vector and exactly the same number of
weights $W_1, \ldots, W_n$.  These are multiplied together and summed. We denote
this as $z$ and call it the \emph{pre-activation} stage of the perceptron.

\[
    z = \sum_{i=1}^{n} W_i x_i = W^{T}x
\]

We can rewrite this as the inner product of $W$ and $x$. The \emph{inner
product} is a way to multiply vectors (element-wise) with result of this
multiplication being a scalar. For further discussion on why these are
equivalent see
\href{http://www.sharetechnote.com/html/Handbook_EngMath_Matrix_InnerProduct.html}{this}.

There is another term called \emph{bias} that is a constant. We can incorporate it
into the weight vector as element $x_0 = 1$ for all our inputs. The
pre-activation stage then becomes

\[
    z = \sum_{i=0}^{n} W_i x_i
\]

This bias term will make more sense when writing code.

Next, comes the non-linear \emph{activation function}, $\sigma$.

\[
    \sigma(a) =
    \begin{cases}
        1, & a \ge 0\\
        0, & a < 0
    \end{cases}
\]

Uniting the pre-activation stage and the activation step gives us the
mathematical model Rosenblatt created for modelling a single biologic neuron:

\[
    y(x) = \sigma(z) =  \sigma(W^{T}x)
\]

$\smiley$

\section{Perceptron Power}

The output of a perceptron is binary, so it can be used for \emph{binary
classification} e.g. that input belongs in one of two classes. But what
constraints are there on the types of classes that Perceptrons can discriminant?

Look back at $z$ with the bias term.

\[
    z = \sum_{i=0}^{N} W_i x_i + b
\]

What does that look like?

\subsection{Perceptron Limitations}

Perceptrons can only discriminant classes that are \emph{linearly separable}.
This means there is a line (or plane) that can linearly divide up the two
classes. See examples \href{https://en.wikipedia.org/wiki/Linear_separability}{here}.

However, the \emph{perceptron convergance theorem} states that if there exists
a linearly separatable solution then the perceptron learning algorithm is
guaranteed to find an exact solution in a finite number of steps.

\end{document}

